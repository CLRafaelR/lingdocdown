%---------------------------------------------------
%
% フォント指定
% 
% なお,アルファベットの設定はYAMLセクションでも可能
%
% 本テンプレートで紹介したフォントをインストールしていない場合は
% 必ず (1) 下記をコメントアウトするか,
% (2) YAMLセクションのincludes: in_header:以下の
% character-config.texをコメントアウトし
% 下記内容を参照しないようにすること
% 
%---------------------------------------------------

\usepackage{fontspec}

%% アルファベットの設定 %%

%%%% ローマン体(論文本文用) %%%%

%\setmainfont[Scale = 1]{TeXGyreTermes}

%%%% サンセリフ体(論文章節タイトル用) %%%%

%\setsansfont[Scale = 1]{TeXGyreHeros}

%%%% タイプライタ体(コードブロック用) %%%%

%\setmonofont[Scale = MatchLowercase]{zcoN}

\usepackage{zxjatype}

%%  非アルファベットの設定  %%
%%                    %%
%% 例:日本語で使うフォント %%

%% このテンプレートは,LaTeXが認識できる日本語フォントのインストールについて
%% 言及していないので,下記はコメントアウトしてある

%%%% 明朝体(論文本文用) %%%%

% \setjamainfont[BoldFont = SourceHanSansJP-Bold.otf]{SourceHanSerifJP-Light.otf}

%%%% サンセリフ体(論文章節タイトル用) %%%%

% \setjasansfont[Scale=1, BoldFont = SourceHanSansJP-Bold.otf]{SourceHanSansJP-Normal.otf}

%%%% タイプライタ体(コードブロック用) %%%%

% \setjamonofont{SourceHanSansJP-Normal.otf}

%---------------------------------------------------
%
% 行頭インデント・行間の設定
%
%---------------------------------------------------

%\usepackage{indentfirst}
%\setlength{\parindent}{12pt} %日本語用
%%\setlength{\parindent}{1.27cm} %APA 6版 対応
%\parskip=0pt
%
%\renewcommand{\baselinestretch}{1.2}
%
%\usepackage{ragged2e}
%
%\usepackage{calc}

