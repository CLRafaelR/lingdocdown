%---------------------------------------------------
%
% 図表に関する指定
%
%---------------------------------------------------

%% 表が版面の幅に合うように自動で調整 %%

\usepackage{adjustbox} 

%% 欧文誌でよく見られる表にする %%

\usepackage{booktabs}

%% 表のセル内改行を可能にする %%

\usepackage{tabularx}

%% 版面における図表の位置を自動で調整する %%

\usepackage{float} % Don't delete after 2020.06.08 update to rmarkdown 2.2

%%%% 図表の出力位置を指定した場合,必ずその位置に指定する %%%%

\floatplacement{figure}{H}
\floatplacement{table}{H}

%% 図表のキャプション設定 %%

%\usepackage[font=small,labelfont={bf,sf}]{caption} %論文用
%\usepackage[font=tiny,labelfont={bf,sf}]{caption} %beamer用

%beamerでのcaptionフォントサイズ変更は\setbeamerfont{caption}{size=\tiny}で行う

%\captionsetup[table]{skip=-5truemm}
%\captionsetup[table]{belowskip = -7truemm}
%\captionsetup{skip = 2pt, aboveskip = 0pt, belowskip = 0pt}


%% 図を並列した場合に出来る「1a」等のキャプションの設定 %%

%\usepackage[font=footnotesize]{subfig} %論文用
\usepackage[font=tiny]{subfig} %beamer用

%% 英文での図表キャプションを略記する場合 %%

%%%% FigureからFig.へ %%%%
%\addto\captionsenglish{\renewcommand{\figurename}{Fig.}}

%%%% TableからTab.へ %%%%
%\addto\captionsenglish{\renewcommand{\tablename}{Tab.}}

%% 図表本体と図表キャプションの間隔を調整する場合 %%

%%%% 図本体と図キャプションの間隔 %%%%

\addtolength{\abovecaptionskip}{-1truemm}

%%%% 表本体と表キャプションの間隔 %%%%

\captionsetup*[table]{skip = 0pt, belowskip = 0pt}
% \addtolength{\belowcaptionskip}{-1truemm}

%---------------------------------------------------
%
% 囲み記事を作る際の指定
%
%---------------------------------------------------

%% tcolorboxを使う %%

\usepackage{tcolorbox}

%%%% 囲み記事が複数ページをまたげるようにする %%%%

\tcbuselibrary{breakable}

%% ascmacパッケージを使う %%

\usepackage{ascmac}

